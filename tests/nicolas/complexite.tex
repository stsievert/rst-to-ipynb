\section{Option Algèbre et Calcul Formel de l'Agrégation de
Mathématiques: Tris et complexité}

\subsection{Introduction à la complexité}

\subsubsection{Quelques problèmes}

\begin{itemize}
\item
  Quelle est le meilleur algorithme pour trouver un nom dans un annuaire
  ?
\item
  Quel est la meilleure méthode pour calculer le déterminant d'une
  matrice ?
\item
  Comment prédire le temps que va mettre un programme pour s'exécuter?
\item
  Comment savoir, entre deux algorithmes, lequel est le plus efficace?
\item
  Comment savoir si un algorithme est optimal?
\item
  Comment déterminer si un problème est insoluble en pratique?
\end{itemize}

\subsubsection{Complexité d'un algorithme}

\paragraph{Exemple: recherche naïve dans une liste}

Je recherche le nom «Zorro» dans un annuaire en utilisant la méthode
suivante:

\begin{enumerate}
\item
  Je pars du début de l'annuaire;
\item
  Je compare le nom avec «Zorro»;
\item
  Si oui, j'ai terminé;
\item
  Si non, je recommence en 2 avec le nom suivant.
\end{enumerate}

\textbf{Problème}

Combien est-ce que cela va me prendre de temps?

\paragraph{Synthèse}

On s'est donné un \emph{problème} (rechercher un mot dans un
dictionnaire), un \emph{algorithme} pour le résoudre (recherche
exhaustive). Puis on a introduit un \emph{modèle de calcul}:

\begin{enumerate}
\item
  Choix de la mesure de \emph{la taille d'une instance du problème} (le
  nombre de mots d'un dictionnaire donné)
\item
  Choix des \emph{opérations élémentaires} (comparer deux mots)
\end{enumerate}

Dans ce modèle, on a cherché le nombre d'opérations élémentaires
effectuées par l'algorithme pour un problème de taille \$n\$. C'est ce
que l'on appelle la \emph{complexité de l'algorithme}.

En fait, on a vu deux variations:

\begin{enumerate}
\item
  Complexité au pire (\$n\$ opérations)
\item
  Complexité en moyenne (\$frac\{n\}\{2\}\$ opérations)
\end{enumerate}

À partir de cette information, et en connaissant le temps nécessaire
pour de petites instances du problème, on peut évaluer le temps
nécessaire pour résoudre n'importe quelle instance du problème.

\paragraph{Complexité en mémoire}

La même stratégie peut s'appliquer à toutes les autres
\emph{ressources}. En particulier la \emph{complexité en mémoire}:
combien faut-il de mémoire pour exécuter l'algorithme (au pire, en
moyenne, \ldots{}).

\textbf{Exemple}

Un algorithme a une complexité en mémoire de \$n\^{}2\$. Que peut-on
dire de sa complexité en temps?

\paragraph{Exercices}

\textbf{Exercice}

Donner des algorithmes et leur complexité au pire et en moyenne pour les
problèmes suivants:

\begin{enumerate}
\item
  Calculer la somme de deux matrices carrées
\item
  Calculer le produit de deux matrices carrées
\item
  Rechercher un élément dans une liste
\item
  Calculer le plus grand élément d'une liste

  \begin{quote}
\begin{verbatim}
sage: def plus_grand_element(liste):
...       r"""
...       Renvoie le plus grand élément de la liste
...       EXAMPLES::
...           sage: plus_grand_element([7,3,1,10,4,10,2,9])
...           10
...           sage: plus_grand_element([7])
...           7
...       """
...       resultat = liste[0]
...       for i in range(1, len(liste)):
...           # Invariant: resultat est le plus grand element de liste[:i]
...           assert resultat in liste[:i]
...           assert all(resultat >= x for x in liste[:i])
...           if liste[i] > resultat:
...               resultat = liste[i]
...       return resultat

sage: plus_grand_element([7,3,1,10,4,10,2,9])
10
\end{verbatim}

  \begin{quote}
  \textbf{note}

  Digression: invariants, preuve et test
  \end{quote}
  \end{quote}
\item
  Rechercher un élément dans une liste triée
\item
  Insérer un élément dans une liste triée
\end{enumerate}

\subsubsection{Ordres de grandeur}

\paragraph{Exemple: recherche dichotomique}

\paragraph{Quelques courbes de complexité}

\begin{verbatim}
sage: var('n')
n
sage: funs = [n^0, log(n, 10), sqrt(n), n, 50*n, n*(log(n,10)), n^2, n^2.3727, n^log(7,2), n^3, 2^n, 10^n, factorial(n)]
sage: colors = rainbow(len(funs))
sage: def time_label(s, t): return text(s, (1,t), horizontal_alignment = "left")
sage: time_labels = sum(time_label(t,s)
...                     for t,s in [["seconde", 1], ["minute", 60], ["jour",24*3600], 
...                                 ["annee",365*24*3600], ["siecle",100*365*24*3600],["age de l'univers",14*10^9*365*24*3600]])
sage: p = sum( plot(f/10^9,
...             xmin=1, xmax=(100 if f(n=100)>10^19 else 10^10),
...             ymax=10^20,
...             scale="loglog",
...             color=color, legend_label=repr(f))
...        for f,color in zip(funs, colors)) + time_labels
sage: p.show()
\end{verbatim}

\paragraph{Synthèse}

La plupart du temps, il suffit d'avoir un ordre de grandeur du nombre
d'opérations: les constantes sont sans grande importance. Un algorithme
en \$1000log\_\{2\}n+50\$ sera meilleur qu'un algorithme en
\$frac\{n\}\{1000\}\$ dès que l'on s'intéressera à des instances
suffisamment grandes.

Mais voir aussi {[}CTFM1993{]}\_!

\textbf{Définition}

Soient \$f\$ et \$g\$ deux fonctions de \$mathbb\{N\}\$ dans
\$mathbb\{N\}\$ (par exemple les complexités de deux algorithmes).

On note \$f=O(g)\$ si, asymptotiquement, \$f\$ est au plus du même ordre
de grandeur que \$g\$; formellement: il existe une constante \$a\$ et un
entier \$N\$ tels que \$f(n)leq ag(n)\$ pour \$ngeq N\$.

On note \$f=o(g)\$ si, assymptotiquement, \$f\$ est négligeable devant
\$g\$; formellement: pour toute constante \$a\$ il existe \$N\$ tel que
\$f(n)leq ag(n)\$ pour \$ngeq N\$.

\textbf{Proposition}

Quelques règles de calculs sur les \$O()\$:

\begin{enumerate}
\item
  \$O(4n+3)=O(n)\$
\item
  \$O(log n)+O(log n)=O(log n)\$
\item
  \$O(n\^{}\{2\})+O(n)=O(n\^{}\{2\})\$
\item
  \$O(n\^{}\{3\})O(n\^{}\{2\}log n)=O(n\^{}\{5\}log n)\$
\end{enumerate}

\paragraph{Exercices}

\textbf{Exercice}

Simplifier les ordres de grandeurs suivants:

\begin{enumerate}
\item
  \$o(17 n + 124) + o(n) + 3n\$
\item
  \$o(n\^{}2) + 10\^{}\{12\}o(n)\$
\item
  \$n\^{}2 o(n) o(log n)\$
\item
  \$O(n) + o(nlog n)\$
\item
  \$O(n) * o(nlog n)\$
\item
  \$O(nlog n) + o(nlog n)\$
\end{enumerate}

\textbf{Exercice}

Donner des algorithmes et leur complexité au pire et en moyenne pour les
problèmes suivants:

\begin{enumerate}
\item
  Effectuer un pivot de Gauss sur une matrice

  \begin{quote}
  \begin{quote}
  \textbf{note}

  Digression: Complexité arithmétique versus complexité binaire
  \end{quote}
  \end{quote}
\item
  Calculer le déterminant d'une matrice
\end{enumerate}

\subsubsection{Complexité d'un problème}

\textbf{Exercice}

On a vu dans un exercice précédent un algorithme pour rechercher le plus
grand élément d'une liste de nombres et on a obtenu sa complexité.

\begin{enumerate}
\itemsep1pt\parskip0pt\parsep0pt
\item
  Existe-t'il un meilleur algorithme ?
\end{enumerate}

\textbf{Définition}

La \emph{complexité d'un problème} est la complexité du meilleur
algorithme pour le résoudre.

On dit qu'un algorithme est \emph{optimal} si sa complexité coïncide
avec celle du problème.

\textbf{Exercices}

\begin{enumerate}
\item
  Les algorithmes vus précédemment sont-ils optimaux?
\item
  Démontrer que la recherche d'un élément dans une liste triée de taille
  \$n\$ est un problème de complexité \$O(log n)\$.
\item
  On dispose d'un ordinateur pouvant exécuter \$10\^{}\{9\}\$ opérations
  élémentaires par seconde (1GHz). On a un problème (par exemple,
  chercher un mot dans une liste, calculer le déterminant d'une
  matrice), et des instances de taille \$1,10,100,1000\$ de ce problème.
  Enfin, on a plusieurs algorithmes pour résoudre ce problème, dont on
  connaît les complexités respectives: \$O(log n)\$, \$O(n)\$, \$O(nlog
  n)\$, \$O(n\^{}\{2\})\$, \$O(n\^{}\{3\})\$, \$O(n\^{}\{10\})\$,
  \$O(2\^{}\{n\})\$, \$O(n!)\$, \$O(n\^{}\{n\})\$. Évaluer dans chacun
  des cas le temps nécessaire.
\end{enumerate}

\subsection{Comparaison de la complexité de quelques algorithmes de tri}

On a une liste que l'on veut trier, mettons \${[}7,8,4,2,5,9,3,5{]}\$.

\subsubsection{Quelques algorithmes de tri}

\paragraph{Tri sélection}

\begin{enumerate}
\item
  On échange le premier élément avec le plus petit des éléments:
  \$2,8,4,7,5,9,3,5\$
\item
  On échange le deuxième élément avec le plus petit des éléments
  restants: \$2,3,4,7,5,9,8,5\$
\item
  Etc.
\item
  Au bout de \$k\$ étapes, les \$k\$ premiers éléments sont triés; on
  échange alors le \$k+1\$-ième élément avec le plus petit des éléments
  restants.
\item
  À la fin, la liste est triée: \$2,3,4,5,5,7,8,9\$.
\end{enumerate}

\paragraph{Tri fusion}

\begin{enumerate}
\item
  On groupe les éléments par paquets de deux, et on trie chacun de ces
  paquets: \$(7,8),(2,4),(5,9),(3,5)\$.
\item
  On groupe les éléments par paquets de quatres, et on trie chacun de
  ces paquets: \$(2,4,7,8),(3,5,5,9)\$.
\item
  \ldots{}
\item
  Au bout de \$k\$ étapes, les paquets de \$2\^{}\{k\}\$ éléments sont
  triés; on les regroupe par paquets de \$2\^{}\{k+1\}\$ que l'on trie.
\item
  À la fin, tous les éléments sont dans le même paquet et sont triés:
  \$(2,3,4,5,5,7,8,9)\$.
\end{enumerate}

\paragraph{Tri rapide}

\begin{enumerate}
\item
  On choisit une valeur \$p\$ dans la liste que l'on appelle pivot
\item
  On fait des échanges judicieux jusqu'à ce que toutes les valeurs
  strictement plus petites que \$p\$ soient placées avant \$p\$, et les
  valeurs plus grandes soient placées après.
\item
  On applique récursivement l'algorithme sur les éléments avant et après
  \$p\$.
\end{enumerate}

\paragraph{Tri insertion, tri par arbre binaire de recherche}

\subsubsection{Analyse de complexité}

\textbf{Problèmes}

Quelle est le meilleur algorithme de tri?

Les algorithmes de tris en \$O(nlog n)\$ sont ils optimaux?

\textbf{Théorème}

Le tri d'une liste de taille \$n\$ est un problème de complexité
\$O(nlog n)\$.

\textbf{Exercices}

Évaluer au mieux la complexité des problèmes suivants:

\begin{enumerate}
\item
  Calcul du \$n\$-ième nombre de Fibonacci;
\item
  Calcul du déterminant d'une matrice;
\item
  Calcul du rang d'une matrice;
\item
  Calcul de l'inverse d'une matrice;
\item
  Calcul d'un vecteur \$x\$ solution de \$Ax=b\$, où \$A\$ est une
  matrice et \$b\$ un vecteur;
\item
  Calcul du pgcd de deux nombres;
\item
  Test de primalité de \$n\$;
\item
  Recherche du plus court chemin entre deux stations de métro à Paris;
\item
  Calcul de la \$n\$-ième décimale de \$sqrt\{2\}\$;
\item
  Calcul de l'inverse d'un nombre modulo \$3\$;
\item
  Recherche d'un échec et mat en \$4\$ coups à partir d'une position
  donnée aux échecs.
\item
  Problème du sac-à-dos: étant donné un ensemble d'objets de hauteur et
  de poids variables, et un sac à dos de hauteur donnée, charger au
  maximum le sac-à-dos?
\end{enumerate}

\subsection{Travaux pratiques}

\subsubsection{Mesure de temps d'exécution de fonctions existantes}

\textbf{Exercice: complexité de l'algorithme de tri de Python}

\begin{enumerate}
\item
  Téléchargez le fichier \$tris.py
  \textless{}http:../\_images/tris.py\textgreater{}\$\_, et
  sauvegardez-le dans votre répertoire principal (par exemple). Ouvrez
  le avec votre éditeur de texte favori, et consultez le contenu.

  \includegraphics{../media/tris.py}

  Vous pouvez charger le fichier dans Sage avec:

\begin{verbatim}
runfile ~/tris.py
\end{verbatim}

  Dans l'interface bloc note il faut utiliser:

\begin{verbatim}
load ~/tris.py
\end{verbatim}

  Variante: utiliser :func:\$attach\$:

\begin{verbatim}
attach ~/tris.py
\end{verbatim}

  qui recharge le fichier automatiquement à chaque modification.
\item
  La fonction sorted peut être utilisée en Python pour trier une liste.
  En utilisant la fonction \texttt{temps\_moyen\_tri} du fichier fourni,
  tracez la courbe du temps moyen d'exécution de sorted sur des
  permutations, en fonction de leur taille n.

  Indication:

  \begin{itemize}
  \item
    Voir point pour afficher un nuage de points. Que fait l'exemple
    suivant?:

\begin{verbatim}
sage: point( [ [i, i^2] for i in range(10) ] )
\end{verbatim}
  \end{itemize}
\item
  Essayez de trouver, graphiquement ou par régression linéaire avec
  find\_fit, une formule approximative donnant ce temps d'exécution en
  fonction de n.
\item
  Estimer de même la complexité de la fonction suivante:

\begin{verbatim}
sage: def fusion(l1, l2):
...       return sorted(l1+l2)
\end{verbatim}
\end{enumerate}

\begin{quote}
lorsque elle est appliquée à des listes aléatoires, respectivement
triées. Pour cela, implanter une variante \texttt{temps\_moyen\_fusion}
de \texttt{temps\_moyen\_tri}.

Qu'en déduisez vous?

Pour en savoir plus: {[}TimSort{]}\_
\end{quote}

\subsubsection{Première étude pratique de complexité}

\textbf{Exercice}

Implanter la fonction \texttt{recherche} dans le fichier
\texttt{tris.py}.

Estimer sa complexité pratique. Pour cela implanter une variante
\texttt{temps\_moyen\_recherche} de \texttt{temps\_moyen\_tri}.

Variantes pour aller plus loin:

\begin{itemize}
\item
  Utiliser randint pour utiliser des listes aléatoires comme
  échantillons plutôt que des permutations aléatoires.
\item
  Introduire une variable globale \texttt{compteur}, l'incrémenter à
  chaque opération «élémentaire» dans vos fonctions de tris, et
  implanter une fonction \texttt{complexite\_moyenne\_recherche} qui
  utilise cette variable pour mesurer le nombre moyen d'opérations
  élémentaires effectuées.
\item
  Évaluer à la fois la complexité moyenne et la complexité au pire.
\end{itemize}

\textbf{Exercice:}

Même exercice précédement, mais en supposant que les listes sont triées
et utilisant une recherche dichotomique.

Indications: Voir \texttt{while}, sort, et utiliser deux bornes
\texttt{inf} et \texttt{sup}, vérifiant à chaque étape l'invariant
\texttt{inf \textless{}= i \textless{} sup}, où \texttt{i} est la
première position (éventuelle) de \texttt{valeur} dans la
\texttt{liste}.

Comparer les deux courbes. Évaluer la taille maximale d'une liste dans
laquelle on peut faire une recherche en moins d'une heure et en moins
d'une semaine.

\subsubsection{Exercice: Implantation et analyse de quelques algorithmes
de tri}

\textbf{Implantation et analyse de tris}

Implanter des fonctions de tri utilisant chacun des algorithmes qui
suivent. On mettra ces fonctions dans le fichier \texttt{tris.py}, avec
documentation, tests et invariants. Pour vérifier que les tests passent,
lancer depuis un terminal:

\begin{verbatim}
sage -t tris.py
\end{verbatim}

Pour chacune des fonctions de tri, tracer une courbe de temps
d'exécution en moyenne (au pire). Comparer avec les courbes théoriques.

\begin{enumerate}
\item
  Tri à bulle en place.

  Indication: \emph{choisir au préalable le bon invariant!}
\item
  Tri fusion.

  Indications:

  \begin{itemize}
  \item
    Implanter, documenter et tester une fonction pour fusionner deux
    listes triées
  \item
    Utiliser une fonction récursive; au besoin, s'entraîner en
    implantant au préalable une fonction récursive pour calculer n!
  \end{itemize}
\item
  Tri rapide en place.

  Indication: \emph{choisir au préalable le bon invariant!}
\item
  Tri par tas.

  Indication: utiliser le module
  \href{http://docs.python.org/library/heapq.html}{heapq}
\item
  Tri par insertion dans un arbre binaire de recherche (équilibré ou
  non).

  Indications:

  \begin{itemize}
  \item
    Pour construire un arbre binaire étiqueté, on peut utiliser la
    classe LabelledBinaryTree:

\begin{verbatim}
sage: LT = LabelledBinaryTree
sage: t = LT( [ LT([], label=1),  LT([ LT([], label=3), LT([], label=5) ], label=4) ], label=2 )
sage: t
2[1[., .], 4[3[., .], 5[., .]]]
\end{verbatim}

    Pour visualiser un tel arbre:

\begin{verbatim}
sage: view(t, tightpage=True)
\end{verbatim}
  \item
    Définir une fonction récursive \texttt{insert(arbre, i)} qui insère
    un nombre \texttt{i} dans un arbre binaire de recherche. Bien
    déterminer l'invariant qui est préservé par cette fonction.
  \end{itemize}
\end{enumerate}

\subsection{Quelques références}

\begin{itemize}
\itemsep1pt\parskip0pt\parsep0pt
\item
  Wikipédia Française:
  \href{http://fr.wikipedia.org/wiki/Complexité_algorithmique}{Complexité
  algorithmique}
\end{itemize}

\begin{itemize}
\itemsep1pt\parskip0pt\parsep0pt
\item
  \href{http://www.lri.fr/~denise/M2Spec/97-98.1/TDSpec6.ps}{Une fiche
  de TP sur les tris}
\end{itemize}
